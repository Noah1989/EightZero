% !TEX encoding = UTF-8 Unicode
\documentclass[10pt]{book}
\usepackage[utf8]{inputenc}
\usepackage[ngerman]{babel}

\usepackage{geometry}
\geometry{a5paper}
\usepackage[parfill]{parskip}
\usepackage{graphicx}
\usepackage{amssymb}

\title{Mikrocomputer}
\author{Katrin Spahn}
\date{\today}

\begin{document}
\maketitle

\chapter{Schneller Einstieg}

Das erste Kapitel soll dir einen ersten Einblick
in die Bedienung, die Funktionsweise
und in den Aufbau des Mikrocomputers geben.

\section{Checkliste}
Bevor du den Mikrocomputer anschließen und einschalten
kannst, solltest du überprüfen, ob du auch alle Dinge
hast, die du zum Arbeiten mit dem Mikrocomputer brauchst.

%Checkliste
\begin{itemize}
\item Mikrocomputer
\item Bildschirm mit VGA-Anschluss
\item Netzteil
\item PS/2-Tastatur oder USB-Tastatur
mit Adapter für PS/2-Schnitt\-stelle
\end{itemize}

Allerdings unterstützt nicht jede USB-Tastatur
einen Adapter für die PS/2-Schnittstelle.
Benutze also am besten eine USB-Tastatur,
wo der Adapter gleich mit dabei ist,
denn dann kannst du dir sicher sein,
dass die Tastatur den Adapter unterstützt.

\section{Anschließen und Einschalten}
Nun hast du alles beisammen, jetzt kann es also losgehen.
Beim näheren Betrachten hast du bestimmt schon bemerkt,
dass der Mikrocomputer mehere Anschlüsse hat.
Die kleine schwarze Buchse links
ist der Anschluss für das Netzteil.
Den runden Stecker des Netzteiles kannst
ohne weiteres Bedenken in die Buchse schieben.
In die lilane Buchse kommt der PS/2-Stecker,
schau dir den Stecker genau an und stecke ihn
ohne Gewalt richtig herum rein.
Zur Orientierung schaue
Abildung \ref{fig:ps2connector} an.
Den schmalen Bildschirmstecker nun noch
in die Buchse rechts stecken
und nun kann der Anschaltkopf
auf der linken Seite betätigt werden.
Auf dem Bilschirm müsste der
Maschinencode-Monitor erscheinen.

\section{Hello World}
Nun gib das Programm aus
Tabelle \ref{tab:helloworld} ein.
Wenn du es fertig abgetippt hast,
sollte ``Hello World'' auf dem Bildschirm erscheinen.
Fange oben links in der ersten Zeile an
und benutze einfach die Pfeiltasten zum Weiterspringen.

\begin{table}
\setlength{\tabcolsep}{3pt}
\texttt{
\begin{tabular}{ r | rrrrrrrrrrrrrrrr }
E0 &  0& 1& 2& 3& 4& 5& 6& 7& 8& 9& A& B& C& D& E& F \\
\hline
 0 & CD&11&07&21&16&E0&FD&21&41&00&CD&32&08&CD&E8&04 \\
 1 & 3E&1B&B9&20&F8&C9&48&65&6C&6C&6F&2C&20&77&6F&72 \\
 2 & 6C&64&21&00 \\
\end{tabular}
}
\caption{``Hello World'' in Maschinencode}
\label{tab:helloworld}
\end{table}

\end{document}